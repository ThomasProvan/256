 \documentclass[12pt]{article}
\usepackage{url,graphicx,tabularx,array,geometry,amsmath,tikz}
\usepackage{algorithm}% http://ctan.org/pkg/algorithms
\usepackage{algpseudocode}% http://ctan.org/pkg/algorithmicx
\usepackage{listings}
\usetikzlibrary{arrows}
\setlength{\parskip}{1ex} %--skip lines between paragraphs
\setlength{\parindent}{0pt} %--don't indent paragraphs
\newenvironment{myindentpar}[1]% %indent whole paragraph when needed
 {\begin{list}{}%
         {\setlength{\leftmargin}{#1}}%
         \item[]%
 }
 {\end{list}}
%-- Commands for header
\renewcommand{\title}[1]{\textbf{#1}\\}
\renewcommand{\line}{\begin{tabularx}{\textwidth}{X>{\raggedleft}X}\hline\\\end{tabularx}\\[-0.5cm]}
\newcommand{\leftright}[2]{\begin{tabularx}{\textwidth}{X>{\raggedleft}X}#1%
& #2\\\end{tabularx}\\[-0.5cm]}
%\linespread{2} %-- Uncomment for Double Space
\begin{document}

\title{ECS 256 - Problem set 2}
\line
Olga Prilepova, Christopher Patton, Alexander Rumbaugh, John Chen, Thomas Provan

1a)\newline
A coin is flipped $k$ times with $p$ probability of heads. For each head, the coin is flipped one additional time (a bonus flip). The number of bonus flips is referred to as $Y$ and the total number of heads $X$\newline
\newline
Var(X) can be found using the Law of Total Variance, and properties of binomial distributions. We will also need to use part of the derivation of EX:
\begin{equation*}
	\begin{aligned}
	E(X|Y)&= E({X - Y} + Y | Y)\\
	  &= E((X-Y)|Y) + E(Y|Y) &(\textrm{by 3.13})\\
	  &= pY + Y &(\textrm{by 3.110})\\
	  &= (1+p)Y\\
	  \\
	Var(X)	&= E[Var(X|Y)] + Var[E(X|Y)]	&(\textrm{by 9.8})\\
			&= E[Var(X|Y)] + Var[(1+p)Y]	&(\textrm{from above})\\
			&= E[Var(X|Y)] + (1+p)^{2}kp(1-p) &(\textrm{by 3.34 and 3.109})\\
			&= E[Yp(l-p)]  + (1+p)^{2}kp(1-p) &(\textrm{by 3.111})\\
			&= kp^{2}(1-p) + (1+p)^{2}kp(1-p) &(\textrm{by 3.103}) \\
			&= kp(1-p)\left(p+(1+p)^2\right)\\
			\textrm{Using p=0.5} \\
			&= k(0.25)(0.5+(1.5^{2}))\\
			&= 0.6875k\\
	\end{aligned}
\end{equation*}

1b)\newline
In the trapped miner example, a miner chooses between three doors with only one leading to safety after $2$ hours. The other two doors lead back to the door room after $3$ and $5$ hours respectively.\newline
\newline
We are interesting the variance of $Y$, the time it takes to escape the mine. We will build upon Ahmed Ahmedin's solution to EY, where N refers to the total attempts needed to escape and $U_{i}$ refers to the time spent traveling on the $i^{th}$ attempt. 
\begin{equation*}
	\begin{aligned}
	Var(Y)	&= E[Var(Y|N)] + Var[E(Y|N)]	&(\textrm{by 9.8})\\
			&= E[Var(Y|N)] + Var[4N-2]	&(\textrm{by 9.16})\\
			&= E[Var(Y|N)] + 16Var[N]	&(\textrm{by 3.34 and 3.41})\\
			&= E[Var(Y|N)] + 16\cdot\frac{1-1/3}{(1/3)^{2}}	&(\textrm{by 3.93})\\
			&= E[Var(U_{1} + U_{2} + ... + U_{n} |N)] + 96\\
			&= E[Var(U_{1}|N) + ... + Var(U_{N-1}|N + Var(U_{N}|N)] + 96 &(\textrm{by 3.51})\\
			&= E[ 1 + 1 + ... 1 + 0 ] + 96\\
			&= E[N-1] + 96\\
			&= E[N] - 1 + 96	&(\textrm{by 3.17})\\
			&= 3 - 1 + 96	&(\textrm{by 3.92})\\
			&= 98\\
	\end{aligned}
\end{equation*}
We know that Var($U_{i}$|N) is independent because the miner's choice of door does not depend of a previous choice. Since we are conditioning this event on there being N attempts, the values of the first N-1 attempts will either be 3 or 5. So the variance of an individual attempt in this case is 1. The variance of the $N^{th}$ attempt is 0 because that attempt always is the same tunnel.

2a) \newline

For a vector $Q$ of random variables $(Q_1,..Q_n)$ we have:
\begin{equation*}
	\begin{aligned}
	Cov(Q)	&= E(QQ') - E(Q)E(Q') &(\textrm{by 13.53})\\
	\end{aligned}
\end{equation*}

Let $Q=Y|X$, where $Y$ is vector valued. 
Then:
\begin{equation*}
	\begin{aligned}
	Cov(Y|X) &= E\big((Y|X)(Y|X)'\big) - E(Y|X)E(Y|X)' &(\textrm{by 13.53})\\
	\end{aligned}
\end{equation*}

Taking expected value of both sides we have:
\begin{equation*}
	\begin{aligned}
	E\big(Cov(Y|X)\big) &= E\Big(E\big((Y|X)(Y|X)'\big) - E(Y|X)E(Y|X)'\Big)\\
	              &= E\Big(E\big((Y|X)(Y|X)'\big)\Big) - E\Big(E(Y|X)E(Y|X)'\Big)\\
	              &= E(YY') - E\Big(E(Y|X)E(Y|X)'\Big) &(\textrm{by Law of Tot. Expect.})\\
	\end{aligned}
\end{equation*}

Now let $Q=E(Y|X)$, where $Y$ is vector valued. 
Then:
\begin{equation*}
	\begin{aligned}
	Cov(E(Y|X)) &= E\big(E(Y|X)E(Y|X)'\big) - E\big(E(Y|X)\big)E\big(E(Y|X)\big)' &(\textrm{by 13.53})\\
	              &= E\big(E(Y|X)E(Y|X)'\big) - E(Y)E(Y)' &(\textrm{by Law of Tot. Expect.})\\
	\end{aligned}
\end{equation*}

Summing up the left sides and the right sides of these 2 equations we get:

\begin{equation*}
	\begin{aligned}
	E\big(Cov(Y|X)\big) + Cov(E(Y|X)) &= E(YY') - E\Big(E(Y|X)E(Y|X)'\Big) \\&+ E\big(E(Y|X)E(Y|X)'\big) - E(Y)E(Y)'\\
	E\big(Cov(Y|X)\big) + Cov(E(Y|X)) &= E(YY') - E(Y)E(Y)' \\
	E\big(Cov(Y|X)\big) + Cov(E(Y|X)) &= Cov(Y) &(\textrm{by 13.53})\\
	\end{aligned}
\end{equation*}

2.b Yet to be done

\end{document}
